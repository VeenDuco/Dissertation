
\pagenumbering{gobble}
% \pagestyle{empty}

\begin{center}
\huge{\textbf{Alternative Information}}
% \huge{\textbf{Alternative Information}}


\Large{Bayesian Statistics, Expert Elicitation and Information Theory in the Social Sciences}

\vspace*{1cm}

\large{\textbf{Alternatieve Informatie}}

\normalsize{Bayesiaanse Statistiek, Expert Elicitatie en Informatie Theorie in de Sociale Wetenschappen}

\vspace*{.3cm}

\normalsize{(met een samenvatting in het Nederlands)}



\vspace*{2cm}

\Large{\textbf{Proefschrift}}

\vspace*{3cm}

\normalsize

ter verkrijging van de graad van doctor aan de \\
Universiteit Utrecht \\
op gezag van de \\
rector magnificus, prof.dr. H.R.B.M. Kummeling, \\
ingevolge het besluit van het college voor promoties \\
in het openbaar te verdedigen op

\vspace*{.5cm}

vrijdag 13 maart 2020 des ochtends te 10.30 uur


\vspace*{1.5cm}

door


\vspace*{1.5cm}

\Large{\textbf{Duco Veen}}
\normalsize

\vspace*{1cm}

geboren op 13 maart 1990

te Zutphen

\end{center}

%%%%%%%%%%%%%%%%%%%%%%%%%%%%%%%%%% newpage

\newpage

\pagestyle{empty}
\textbf{Promotoren:}


Prof.dr. A.G.J. van de Schoot

\textbf{Copromotoren:}

dr. G. Vink

dr. N.E.E. van Loey

\vspace*{\fill}

\noindent The studies in this thesis were funded by the Netherlands Organization for Scientific Research grant NWO-VIDI-452-14-006.

% 

\newpage
%%%%%%%%%%%%%%%%%%%%%%%%%%%%%%%%%% newpage


\textbf{Beoordelingscommissie:}

Prof. dr. R. Geenen

Prof. dr. I.G. Klugkist

Dr. D.L. Oberski 

Prof. dr. S. van der Stigchel

Prof. dr. E.M. Wagenmakers 

\vspace*{\fill}

\noindent Veen, Duco

\noindent ISBN NUMMERS / DRUKKER ETC.

\newpage

\textbf{De Quasi-neutrale Oplossingsfabriek}

\textit{Een tocht door het land van de ivoren torens, waar de objectieve waarheid wordt gemaakt}

De utopische traditie is de afgelopen decennia drastisch uitgedund. Wat ons resteert zijn technotopia's, klimaatdystopia's en nostalgie naar de sixties. Maar als altijd, hebben wij ook vandaag een baken nodig aan de horizon van tijd, om naartoe te koersen en gezamenlijk grote beslissingen te kunnen nemen. Eeuwenlang was de wetenschap de plek bij uitstek waar Utopia ontstond, vooral ten tijde van een intellectuele revolutie. Dus waarom ontstaan er geen utopieën vandaag? We bezoeken het land van de ivoren torens.

Hoe komen we tot kennis? Uit de wereld om ons heen verzamelen we gegevens die wij omhoog brengen naar het hart van de wetenschap, de nok van de ivoren toren waar de waarheid wordt gemaakt. En wanneer een theorie is gemaakt, gaan we checken of het klopt. Want dat is tenslotte wat wij doen op de universiteit, we construeren waarheden. Allereerst zal de kennis door de objectieve waarheidstrechter gaan, waar het wordt gestript van kwalitatieve en subjectieve aspecten en van normen en waarden. Wat er overblijft, neutrale cijfers, gaat naar de binaire pers, klaar voor de computers van de programmeurs. Met algoritmes transformeren zij de eentjes en nulletjes in efficiëntie, snelheid en groei, en schuiven het dan door naar de economen. Zittend op het GNP en maaiend met grote grijpers plaatsen de economen iedereen netjes ergens in de fabriek van onze maatschappij.

In blinde processie bewegen we voort op het tikken van de klok van de ene cel naar de volgende, gevangen in de oneindigheid van onze dagelijkse routines. Billboards moedigen ons aan om geld dat we niet hebben uit te geven aan zooi die we willen, om indruk te maken op de mensen die we eigenlijk niet uit kunnen staan. Rechts van de kloof vinden we de kunstenaars, muzikanten, filosofen, én de utopisten - bestempeld als dromers of extremisten. Voor hen is het te gevaarlijk om over te steken, zij zijn overbodig hier, in een maatschappij van cijfers, feiten en neutrale waarheden.

Eeuwenlang was de wetenschap bij uitstek de plek waar Utopia ontstond, maar vandaag is dat veranderd. De universiteit van de 21ste eeuw is eigenlijk niets meer dan een fabriek, een quasi-neutrale oplossingsfabriek, waar iedereen te druk is met schrijven om nog iets te lezen, en te druk met publiceren voor een debat. Als wij willen dat er waardevolle utopieën kunnen ontstaan binnen de universiteit, dan zal het proces van kennis verwerven en de interpretatie van waarheid misschien wel moeten veranderen...

\vspace*{\fill}

Illustratie op achterzijde en bijbehorend verhaal door:

Carlijn Kingma (https://www.carlijnkingma.com/)

\newpage

%\frontmatter

% \makeatletter
% \renewcommand\chaptermark[1]{%
% 	\markboth{\MakeUppercase{#1}}{}
% }
% \makeatother
% 
% \markboth{CONTENTS}{CONTENTS}

\setcounter{tocdepth}{1}
\tableofcontents
\thispagestyle{empty}
\addtocontents{toc}{\protect\thispagestyle{empty}}
% https://tex.stackexchange.com/questions/2995/removing-page-number-from-toc

